Convolutional networks are at the core of most state-of-the-art
computer vision solutions for a wide variety of tasks.
Since 2014 very deep convolutional networks started to become
mainstream, yielding substantial gains in various benchmarks.
Although increased model size and computational cost tend to
translate to immediate quality gains for most tasks (as long as enough
labeled data is provided for training), computational efficiency and
low parameter count are still enabling factors for various use
cases such as mobile vision and big-data scenarios.
Here we are exploring ways to scale up networks in ways that aim at
utilizing the added computation as efficiently as possible by
suitably factorized convolutions and aggressive regularization.
We benchmark our methods on the ILSVRC 2012 classification challenge
validation set demonstrate substantial gains over the state of the art:
$21.2\%$ top-$1$ and $5.6\%$ top-$5$ error for {\it single frame}
evaluation using a network with a computational cost of $5$ billion
multiply-adds per inference and with using less than 25 million
parameters. With an ensemble of $4$ models and multi-crop
evaluation, we report $3.5\%$ top-$5$ error and $17.3\%$
top-$1$ error.
