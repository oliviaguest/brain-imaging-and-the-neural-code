\section{Experimental Results and Comparisons}
\begin{table}
{\small
 \begin{center}
   \begin{tabular}[H]{|l|c|c|c|}
   \hline
   {\bf Network} & \stackanchor{\bf Top-1}{Error} & \stackanchor{\bf Top-5}{Error} & \stackanchor{\bf Cost}{Bn Ops} \\
   \hline\hline
   GoogLeNet~\cite{szegedy2015going} & 29\% & 9.2\% & 1.5 \\
   \hline
   BN-GoogLeNet & 26.8\% & - & {\bf 1.5} \\
   \hline
   BN-Inception~\cite{ioffe2015batch} & 25.2\% & 7.8 & 2.0 \\
   \hline
   Inception-v2 & 23.4\% & - & 3.8 \\
   \hline
   \shortstack[l]{Inception-v2 \\RMSProp} & 23.1\% & 6.3 & 3.8 \\
   \hline
   \shortstack[l]{Inception-v2 \\ Label Smoothing} & 22.8\% & 6.1 & 3.8 \\
   \hline
   \shortstack[l]{Inception-v2 \\ Factorized $7\times 7$} & 21.6\% & 5.8 & 4.8 \\
   \hline
   \stackanchor{Inception-v2}{BN-auxiliary} & {\bf 21.2}\% & {\bf 5.6}\% & 4.8 \\
   %% \hline
   %% BN-Inception-v3 & 23.7\% & 3.8 & ??? & 3.8 \\
   \hline
   \end{tabular}
 \end{center}
 }
\caption{Single crop experimental results comparing the cumulative effects on
 the various contributing   factors. We compare our numbers with the best
 published single-crop inference for Ioffe at
 al~\cite{ioffe2015batch}. For the ``Inception-v2'' lines, the
 changes are cumulative and each subsequent line includes the new change in
 addition to the previous ones. The last line is referring to all the changes
 is what we refer to as ``Inception-v3'' below. Unfortunately, He et
 al~\cite{he2015delving} reports the only 10-crop evaluation results, but not
 single crop results, which is reported in the Table~\ref{resultsmulticrop}
 below.
}
 \label{results}
\end{table}
Table~\ref{results} shows the experimental results about the recognition
performance of our proposed architecture (Inception-v2) as described in
Section~\ref{revisited}. Each Inception-v2 line shows the result of the
cumulative changes including the highlighted new modification plus all the
earlier ones. Label Smoothing refers to method described in Section~\ref{smoothing}.
Factorized $7\times 7$ includes a change that factorizes the first $7\times 7$
convolutional layer into a sequence of $3\times 3$ convolutional layers.
BN-auxiliary refers to the version in which the fully connected layer
of the auxiliary classifier is also batch-normalized, not just the convolutions.
We are referring to the model in last row of Table~\ref{results} as Inception-v3
and evaluate its performance in the multi-crop and ensemble settings.

All our evaluations are done on the 48238 non-blacklisted examples on the
ILSVRC-2012 validation set, as suggested by ~\cite{russakovsky2014imagenet}.
We have evaluated all the 50000 examples as well and the results were roughly
0.1\% worse in top-5 error and around 0.2\% in top-1
error. In the upcoming version of this paper, we will verify our ensemble
result on the test set, but at the time of our last evaluation of
BN-Inception in spring~\cite{ioffe2015batch} indicates that the test and
validation set error tends to correlate very well.

\begin{table}
{\small
 \begin{center}
   \begin{tabular}[H]{|l|c|c|c|}
   \hline
   {\bf Network} &
   \stackanchor{\bf Crops}{\bf Evaluated} &
   \stackanchor{\bf Top-5}{\bf Error} &
   \stackanchor{\bf Top-1}{\bf Error} \\
   \hline\hline
   GoogLeNet~\cite{szegedy2015going} & 10 & - & 9.15\% \\
   \hline
   GoogLeNet~\cite{szegedy2015going} & 144 & - & 7.89\% \\
   \hline
   VGG~\cite{simonyan2014very} & - & 24.4\% & 6.8\% \\
   \hline
   BN-Inception~\cite{ioffe2015batch} & 144 & 22\% & 5.82\% \\
   \hline
   PReLU~\cite{he2015delving} & 10 & 24.27\% & 7.38\% \\
   \hline
   PReLU~\cite{he2015delving} & - & 21.59\% & 5.71\% \\
   \hline
   Inception-v3 & 12 & 19.47\% & 4.48\% \\
   \hline
   Inception-v3 & 144 & \bf{18.77\%} & \bf{4.2\%} \\
   \hline
   \end{tabular}
 \end{center}
 }
\caption{Single-model, multi-crop experimental results comparing the cumulative effects on
 the various contributing   factors. We compare our numbers with the best
 published single-model inference results on the ILSVRC 2012 classification
 benchmark.} \label{resultsmulticrop}
\end{table}

\begin{table}
{\small
 \begin{center}
   \begin{tabular}[H]{|l|c|c|c|c|}
   \hline
   {\bf Network} &
   \stackanchor{\bf Models}{\bf Evaluated} &
   \stackanchor{\bf Crops}{\bf Evaluated} &
   \stackanchor{\bf Top-1}{\bf Error} &
   \stackanchor{\bf Top-5}{\bf Error} \\
   \hline\hline
   VGGNet~\cite{simonyan2014very} & 2 & - & 23.7\% & 6.8\% \\
   \hline
   GoogLeNet~\cite{szegedy2015going} & 7 & 144 & - & 6.67\% \\
   \hline
   PReLU~\cite{he2015delving} & - & - & - & 4.94\% \\
   \hline
   BN-Inception~\cite{ioffe2015batch} & 6 & 144 & 20.1\% & 4.9\% \\
   \hline
   Inception-v3 & 4 & 144 & {\bf 17.2\%} & {\bf 3.58\%}$^*$ \\
   \hline
   \end{tabular}
 \end{center}
}
\caption{Ensemble evaluation results comparing multi-model, multi-crop
reported results. Our numbers are compared with the best  published
ensemble inference results on the ILSVRC 2012 classification benchmark.
$^*$All results, but the top-5 ensemble result reported are 
on the validation set. The ensemble yielded $3.46\%$ top-5 error on the
validation set.}
\label{resultsensemble}
\end{table}


%% \section{Qualitative Analysis of the Improvements}

%% Here we analyze the gains and losses with respect to the original GoogLeNet
%% submission and one state-of-the-art VGG network.

%% TODO(Generate classes with most gains/losses)
%% TODO(Examples of gains/losses on ImageNet)
%% TODO(Ensemble results with 4 or 5 of the last generation Inception on ImageNet)
